%%%%%%%%%%%%%%%%%%%%%%%%%%%%%%%%%%%%%%%%%
% Medium Length Professional CV
% LaTeX Template
% Version 2.0 (8/5/13)
%
% This template has been downloaded from:
% http://www.LaTeXTemplates.com
%
% Original author:
% Trey Hunner (http://www.treyhunner.com/)
%
% Important note:
% This template requires the resume.cls file to be in the same directory as the
% .tex file. The resume.cls file provides the resume style used for structuring the
% document.
%
%%%%%%%%%%%%%%%%%%%%%%%%%%%%%%%%%%%%%%%%%

%----------------------------------------------------------------------------------------
%	PACKAGES AND OTHER DOCUMENT CONFIGURATIONS
%----------------------------------------------------------------------------------------

\documentclass{resume} % Use the custom resume.cls style

\usepackage[left=0.75in,top=0.6in,right=0.75in,bottom=0.6in]{geometry} % Document margins
\usepackage{hyperref}
\usepackage{graphicx}
\usepackage{multicol}

%----------------------------------------------------------------------------------------
%	PERSONAL DATA
%----------------------------------------------------------------------------------------\\

\name{Giovanni Ciatto}
\phone{(+39) 370 10 84 490}
\skype{giovanni.ciatto}
\birthday{January 4, 1992} 
\genre{Male}
\country{Italy}
%\address{Via Bagnile, 67, 47522 Cesena (FC), Italy}
\address{Cesena (FC), Italy}
\email{\texttt{giovanni.ciatto@unibo.it}}
\www{\url{https://about.me/gciatto}}
\photo{mericci.jpg}

\begin{document}

\begin{rSection}{Desired occupation}
A researcher position within the fields of self-organizing \emph{distributed systems}, with a particular focus on the models, architectures, infrastructures, and technologies enabling the coordination of massive distributed computations.
\end{rSection}

%----------------------------------------------------------------------------------------
%	EDUCATION SECTION
%----------------------------------------------------------------------------------------

\begin{rSection}{Education}
\begin{rSubsection}{PhD Student}{\textbf{November, 2017 $\rightarrow$ Now}}{Data Science and Computation}{University of Bologna, Italy}
	\item Studies in the field of Data Science and Big Data, focusing on novel enabling approaches, architectures and technologies.
\end{rSubsection}

% Dissertation: 16/3/2017
\begin{rSubsection}{Master's degree}{\textbf{October, 2014 $\rightarrow$ March, 2017}}{Computer Science and Engineering}{University of Bologna, Italy}
\item Studies in the field of programming paradigms, computational models, distributed systems coordination, robotics, machine learning and artificial vision, web applications, business intelligence, model-driven software engineering

\item 110/110 \emph{cum laude}
\end{rSubsection}
\begin{rSubsection}{Master's degree thesis}{\textbf{March 16, 2017}}{Third generation neural networks: formalization as timed automata, validation and learning}{\begin{flushright}
			Relator: Prof. Gianluigi Zavattaro
		\end{flushright}}
	\item Co-relators: Elisabetta De Maria, Cinzia Di Giusto (University of Nice)
	\item Modeling of spiking neural networks by means of the ``timed automata'' formalism, validation of the so-achieved model and proposal of a learning approach.
	
	\item External site: \url{https://github.com/gciatto/snn_as_ta}
\end{rSubsection}
% Dissertation: 9/10/2014
\begin{rSubsection}{Bachelor's degree}{\textbf{September, 2011 $\rightarrow$ October, 2014}}{Electronics, informatics and telecommunications engineering}{University of Bologna, Italy}
%	{University of Bologna, Seconda Facolt\`a di Ingegneria, Cesena, Italy}%	
	\item Studies in the field of networking, signal processing, software engineering and programming languages.
	\item 110/110 \emph{cum laude}
\end{rSubsection}

\begin{rSubsection}{Bachelor's degree thesis}{\textbf{October 9, 2014}}{Impiego combinato di GPS, BLE e riconoscimento di immagini per individuare entità nella realtà aumentata}{Relator: Prof. Mirko Viroli}
	
	\item Steering a user perceiving an augmented world by integrating GPS for long distances, bluetooth beacons for indoor localization, and marker recognition for closer objects.
	
	\item External site: \url{http://amslaurea.unibo.it/7658/}
\end{rSubsection}

\begin{rSubsection}{High-school diploma}{\textbf{2006 $\rightarrow$ 2011}}{Scientific curriculum}{Liceo Scientifico ``C. Caminiti'', S. Teresa di Riva (ME), Italy}
	\item 98/100
\end{rSubsection}

\end{rSection}

\clearpage

%----------------------------------------------------------------------------------------
%	PUBLICATIONS SECTION
%----------------------------------------------------------------------------------------

\begin{rSection}{Publications}

\begin{rSubsection}{Paper}{}{Spiking Neural Networks as Timed Automata}{\emph{G. Ciatto, E. De Maria, C. Di Giusto}}
	\item Proc. of the Thematic Research School on Advances in Systems and Synthetic Biology (ASSB), EDP Sciences, (2017).
\end{rSubsection}

\begin{rSubsection}{Paper}{\textbf{Accepted on June 2017}}{Programming the Interaction Space Effectively with ReSpecTX}{\emph{G. Ciatto, S. Mariani, A. Omicini}}
	\item (submitted to) IDC 2017: $11^{th}$ International Symposium on Intelligent Distributed Computing
\end{rSubsection}

%\begin{rSubsection}{Article}{\textbf{Submitted on May 2017}}{Parameter Learning for Spiking Neural Networks modelled as Timed Automata}{\emph{G. Ciatto, E. De Maria, C. Di Giusto}}
%	\item (submitted to) CMSB 2017: 15th Conference on Computational Methods in Systems Biology
%\end{rSubsection}

\begin{rSubsection}{Extended Abstract}{}{Novel Opportunities for Tuple-based Coordination: XPath, the Blockchain, and Stream Processing}{\emph{S. Mariani, A. Omicini, G. Ciatto}}
	\item WOA 2017: 18th Workshop "From Objects to Agents", pp. 61-64
\end{rSubsection}

\end{rSection}

%----------------------------------------------------------------------------------------
%	LANGUAGE SECTION
%----------------------------------------------------------------------------------------

\begin{rSection}{Language}
\begin{center}
\begin{tabular}{|c|c|c|c|c|c|}
	\hline
	&\textbf{Listening}&\textbf{Reading}&\textbf{Interaction}&\textbf{Speaking}&\textbf{Writing}\\\hline
	\textbf{Italian}&\multicolumn{5}{c}{Native language}\vline\\\hline
	\textbf{English}&C1&C1&C1&C1&C1\\\hline
	\textbf{French}&A1&A1&A1&A1&A1 \\
	\hline
\end{tabular}
%\item \textit{Skills acquired during Internship in Sophia-Antipolis (France)}
\end{center}
%\begin{rSubsection}{Certifications}{}{}{}
%	\item \textbf{IELTS}, British Council, European level: C1 \hfill \textbf{July, 2015}
%\end{rSubsection}
\end{rSection}


%----------------------------------------------------------------------------------------
%	WORK EXPERIENCE SECTION
%----------------------------------------------------------------------------------------

\begin{rSection}{Experience}
	
\begin{rSubsection}{Teaching assistant for the course ``Distributed Systems''}{\textbf{September, 2017 $\rightarrow$ Now}}{School of Engineering and Architecture}{University of Bologna, Italy}
	\item Distributed architectures. Agent-based technologies and middlewares. Computational logic, logic programming and Prolog.
\end{rSubsection}
	
\begin{rSubsection}{Temporary Research Fellow, ``Assegnista di ricerca''}{\textbf{April 24, 2017 $\rightarrow$ October 31, 2017}}{Department of Informatics -- Science and Engineering (DISI)}{University of Bologna, Italy}
	\item Project title: ``Language and platform techniques for complex software systems development''
\end{rSubsection}

\begin{rSubsection}{Internship}{\textbf{June, 2016 $\rightarrow$ December, 2016}}{I3S Laboratory}{University of Nice--Sophia Antipolis, France}
	\item Studies in the field of theoretical informatics applied to third generation neural networks
\end{rSubsection}

\begin{rSubsection}{Internship}{\textbf{October, 2013 $\rightarrow$ February, 2014}}{A GUI for the Alchemist Simulator}{APICe lab, University of Bologna, Cesena, Italy}
\item Design and implementation of a GUI for the Alchemist simulator and its integration with OpenStreetMap
\item Supervisor: Prof. Mirko Viroli
\item Alchemist: \url{http://alchemistsimulator.github.io}
\item OpenStreetMap: \url{https://www.openstreetmap.org}
\end{rSubsection}

\end{rSection}

\clearpage

%----------------------------------------------------------------------------------------
%	TECHNICAL STRENGTHS SECTION
%----------------------------------------------------------------------------------------

\begin{rSection}{Technical Strengths}
\begin{tabular}{ @{} >{\bfseries}l @{\hspace{6ex}} l }
Hardware configuration 	& Desktop \& notebook PCs assembling \\
Programming Paradigms	& imperative, object oriented, functional,\\
						& logic, constraint programming\\
Software configuration 	& Windows and Linux installation and configuration\\
Programming Languages 	& Java, Scala, Xtend, C\# \& VB.Net, Haskell, Prolog,\\
						& JavaScript, C, Python, Minizinc, COBOL\\
Protocols \& APIs 		& Socket (TCP \& UDP), HTTP, RESTful WebAPI \\
Databases 				& SQL, PostgreSQL, MySQL, IBM Informix \\
Development tools 		& Git, Mercurial, Maven, Gradle, Swagger, Xtext\\
Markup languages 		& XML, HTML, Markdown, \LaTeX \\
IDEs 					& Eclipse, IntelliJ Idea, Visual Studio, \\
						& Android Studio, PyCharm
\end{tabular}						
\end{rSection}


%----------------------------------------------------------------------------------------
%	ADDITIONAL INFO SECTION
%----------------------------------------------------------------------------------------

\begin{rSection}{Additional information}

\item \textbf{About me}: I am an experienced developer and designer. 
I can both work alone or as a team, with or without my favorite IDEs. 
%I often exploit version control system such as Git or Mercurial to make my projects safe. 
I prefer a \emph{model driven} approach when designing software but I can easily switch my mind to some agile approach, like SCRUM, if needed.

\item \textbf{Interests}: formal models and languages, from both the designer and user point of view; MAS \& coordination within distributed systems; logic or other declarative paradigms; artificial intelligence and machine learning; learning as much languages as possible!
\end{rSection}

%----------------------------------------------------------------------------------------
%	SIGNATURE SECTION
%----------------------------------------------------------------------------------------
\begin{flushright}
	\textbf{Giovanni Ciatto}, \today
	
%	\includegraphics[width=6cm]{./signature.png}
\end{flushright}
%\begin{rSection}{Signature}
%\begin{rSubsection}{Signature}{\today}{}{}
%\end{rSubsection}				
%\end{rSection}
\end{document}